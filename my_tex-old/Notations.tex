\newpage

\addcontentsline{toc}{chapter}{\large{Notations}}
\begin{center}
	\LARGE{\textbf{Notations}}
\end{center}


\begin{eqnarray*}
	\mathbb{N}^* & & \quad \text{Ensemble des nombres entiers naturels privés de 0} \\
	\mathbb{R} & & \quad \text{Ensemble des nombres réels} \\
	\mathbb{C} & & \quad \text{Ensemble des nombres complexes} \\
	\mathbb{K} & & \quad \text{Le  corps} \ \ \mathbb{R} \ \ \text{ou} \ \ \mathbb{C}\\
	A[X] & & \quad \text{Anneau des polynômes à coefficients dans A d'indéterminée X}\\
	\mathcal{M}_n(\mathbb{K}) & & \quad \text{Ensemble des matrices carrées d'ordre n à coefficients dans $\mathbb{K}$}\\
	\mathbb{F}(A) & & \quad \text{Ensemble des filtrations sur le module A}\\
	R(A,f) & & \quad \text{Anneau de Rees par rapport à la filtration $f$ de A}\\
	\mathcal{R}(A,f) & & \quad \text{Anneau de Rees généralisé par rapport à la filtration $f$ de A}\\
	f_{I} & & \quad \text{La filtration I-adique} \\
	e & & \quad \text{Élément neutre d'un groupe} \\
	0_{A} & & \quad \text{Élément neutre par rapport à la première loi de l'anneau} \\
	1_{A} & & \quad \text{Élément neutre par rapport à la deuxième loi de l'anneau} \\
	id_{A} & & \quad \text{L'application identité de A} \\
	P(f) & & \quad \text{Clôture prüférienne de $f$.} \\
	f^{(k)} & & \quad \text{Filtration extraite de $f$ d'ordre $k$} \\
	t_kf & & \quad \text{Filtration tronquée de $f$ d'ordre $k$} \\
	I_{r} & & \quad \text{Matrice identité d'ordre $r$} \\
	0_{r,1} & & \quad \text{Matrice nulle de dimension (r,1)} \\
	P_{T} & & \quad \text{Polynôme caractéristique associé à T} \\
\end{eqnarray*}